%\documentclass[11pt,reqno]{amsart}
\documentclass[11pt,reqno]{article}
\usepackage[margin=.8in, paperwidth=8.5in, paperheight=11in]{geometry}
%\usepackage{geometry}                % See geometry.pdf to learn the layout options. There are lots.
%\geometry{letterpaper}                   % ... or a4paper or a5paper or ... 
%\geometry{landscape}                % Activate for for rotated page geometry
%\usepackage[parfill]{parskip}    % Activate to begin paragraphs with an empty line rather than an indent7
\usepackage{graphicx}
\usepackage{pstricks}
\usepackage{amssymb}
\usepackage{epstopdf}
\usepackage{amsmath}
\usepackage{subfigure}
\usepackage{caption}
\pagestyle{plain}
%\renewcommand{\topfraction}{0.3}
%\renewcommand{\bottomfraction}{0.8}
%\renewcommand{\textfraction}{0.07}
\DeclareGraphicsRule{.tif}{png}{.png}{`convert #1 `dirname #1`/`basename #1 .tif`.png}

\title{Real Analysis $\mathbb{I}$: \\ Assignment 3}
\author{Andrew Rickert}
\date{Started: March 13, 2011 \\ \hspace{1pt} Ended: March ??  2010}                                           % Activate to display a given date or no date

\begin{document}
\maketitle


% Page 1
\begin{flushleft} 
\textbf{Class 18.100B} - Problem 1\\
\rule{500pt}{1pt}\\
\end{flushleft} 

We need to show how the norm can define a metric on $\mathbb{R}$. Suppose we are given the definition of the norm as in the problem. We define the metric to based on the norm as $d(p,q) \equiv \|p-q\|$.\\
\indent We know need to verify that the metric defined in this way satisfies the axioms for a metric.
First we need to show that 
\begin{equation}
d(p,q) > 0 \; \text{if} \; p \neq q \; \text{and} \; d(p,p) = 0 \label{rule1}
\end{equation} 
From rule i) for the norm we know that $d(p,q) = \|p-q\| \ge 0$,  since also we have that $\|x\| = 0 \iff x = 0$ we know that $p-q \ne 0 \implies \|p-q\| \neq 0$. So, if $p \neq q$ then $d(p,q) \neq 0$ which shows $d(p,q) > 0$ and completes the first part of (\ref{rule1}). From the definition of the metric we have $d(p,p) = \|p - p\| = \| 0 \| = 0$ again by rule (i). The completes the second part of (\ref{rule1}).\\
\indent Next we have to show that 
\begin{equation}
d(p,q) = d(q,p) \label{rule2}
\end{equation} 
According to rule (ii) it is true that $\| \lambda x \| = |\lambda| \|x \|$. We perform the following calculation.
\begin{eqnarray*}
d(p,q) & = & \| p - q \| \\
            & = & \| (-1) (q - p) \| \\
            & = & |-1| \|q - p\| \; \text{by rule (ii)} \\
            & = & \| q - p \| = d(q,p)
\end{eqnarray*}
We are given rule (iii) as $\| x+ y \| =  \| x \|  +  \| y \|$. The calculation to verify that $d(p,q) \le d(p,r) + d(r,q)$ goes as follows:
\begin{eqnarray*}
d(p,q) & = & \| p - q \| \\
	  & = & \| p - r + r - q \| \\
	  & = & \| p - r \| + \| r - q \| \; \text{by rule (iii)} \\
	  & = & d(p,r) + d(r,q)
\end{eqnarray*}
This verifies all of the properties of the metric based on the norm defined in the problem.

\newpage
\vspace{15pt}
\begin{flushleft} 
\textbf{Class 18.100B} - Problem 2\\
\rule{500pt}{1pt}\\
\end{flushleft} 

In order to show that $d_1(x,y) = \frac{d(x,y)}{1 + d(x,y)}$ is a metric, given that $d(x,y)$ is, we will use the definition of $d_1(x,y)$ to show the three properties of the metric are satisfied.\\
\indent First we need to $d_1(x,y) > 0$ if $x \neq y$ and $d(x,x) = 0$ otherwise. Now since $d(x,y) > 0$ if $x \neq y$ then $1 + d(x,y) > 1 > 0$ so $1+ d(x,y) > 0$ therefore we have $\frac{d(x,y)}{1 + d(x,y)} > 0$ by the field axioms. Also, because $d(x,x) = 0$ by definition, $d_1(x,x) = \frac{d(x,y)}{1 + d(x,y)} = \frac{0}{1 + 0} = 0$.\\
\indent Next we show that $d_1(x,y)$ is symmetric. This is clear from the following calculation
\begin{eqnarray*}
d_1(x,y) & = & \frac{d(x,y)}{1 + d(x,y)} \\
	      & = & \frac{d(y,x)}{1 + d(y,x)} \quad \text{By the symmetry of $d(x,y)$} \\
	      & = & d_1(y,x)
\end{eqnarray*}
Finally we need to show that $d_1(x,y)$ satisfies the triangle inequality. From the definition of the metric $d$ we have that $d(x,z) \le d(x,y) + d(y,z)$ and similarly $1+ d(x,z) \le 1+ d(x,y) + d(y,z)$. Because $d(p,q) > 0$ for all $p, q$we can divide the first expression by second to get\\
\begin{eqnarray}
d_1(x,y) & = & \frac{d(x,y)}{1 + d(x,y)} \le \frac{d(x,y)+ d(y,z)}{1+ d(x,y) + d(y,z)} \nonumber \\ 
	      & = & \frac{d(x,y)}{1+ d(x,y) + d(y,z)} + \frac{d(y,z)}{1+ d(x,y) + d(y,z)} \label{largeTriangle}
\end{eqnarray}
Now both $0 \le d(x,y)$ and $0 \le d(y,z)$ from the definition of $d(p,q)$ so we have in the first instance  $0 \le d(x,y) \implies 1 + d(y,z) \le 1 + d(x,y) + d(y,z) \implies \frac{1}{1 + d(x,y) + d(y,z)} \le \frac{1}{1 + d(y,z)}$ and by a similar calculation we have $ \frac{1}{1 + d(x,y) + d(y,z)} \le \frac{1}{1 + d(x,y)}$. Using these expressions in (\ref{largeTriangle}) we get the following 
\begin{eqnarray*}
d_1(x,y) & \le & \frac{d(x,y)}{1+ d(x,y) + d(y,z)} + \frac{d(y,z)}{1+ d(x,y) + d(y,z)}  \\
	      & \le  &  \frac{d(x,y)}{1+ d(x,y)} +  \frac{d(y,z)}{1+ d(y,z)} \\
	      & = & d_1(x,y) + d_1(y,z)
\end{eqnarray*}
This completes the demonstration of the third property of the metric.

\vspace{15pt}
\begin{flushleft} 
\textbf{Class 18.100B} - Problem 3\\
\rule{500pt}{1pt}\\
\end{flushleft} 

\noindent Part a) Show $A^\circ \cup B^\circ \subset (A \cup B)^\circ$ \\ \newline
\indent Assume that $x \in A^\circ \cup B^\circ$, this means either $x \in A^\circ$ or $x \in B^\circ$. Suppose that $x \in A^\circ$, then there is a neighborhood containing $N(x)$ such that $N(x) \subset A$ by the definition of $A^\circ$. This means that also $N(x) \subset A \cup B$, that is to say $x \in (A \cup B)^\circ$. The conclusion for the case where $x \in B^\circ$ is similar we have therefore that $A^\circ \cup B^\circ \subset (A \cup B)^\circ$. \newline

\newpage
 \noindent Part b) Show $A^\circ \cap B^\circ = (A \cap B)^\circ$ \\ \newline
 \indent Assume that $x \in A^\circ \cap B^\circ$ then $x \in A^\circ$ and $x \in B^\circ$. From the definition of the interior we have a neighborhood around x in A, $N_{\delta_1}(x) \subset A$ and a neighborhood around x in B, $N_{\delta_2}(x) \subset B$. Since the intersection of two open sets is open and $x \in N_{\delta_1}(x) \cap N_{\delta_2}(x)$ there must be a neighborhood of x in the intersection so $N_\delta(x) \subset N_{\delta_1}(x) \cap N_{\delta_2}(x)$ where $\delta \le \delta_1$ and $\delta \le \delta_2$. This shows that $N_\delta(x) \subset A \cap B$ which is to say that $x \in (A \cap B)^\circ$. Thus $x \in A^\circ \cap B^\circ \subset (A \cap B)^\circ$.\\
\indent Now assume that $x \in (A \cap B)^\circ$. This means there is a neighborhood around x, $N(x)$, such that $N(x) \subset A \cap B$. This means that $N(x) \subset A$ and $N(x) \subset B$,  which by the definition of the interior means $x \in A^\circ$ and $x \in B^\circ$ therefore $x \in A^\circ \cap B^\circ$. Thus, $x \in (A \cap B)^\circ \subset A^\circ \cap B^\circ$ and we have the result.
 
\vspace{15pt}
\begin{flushleft} 
\textbf{Class 18.100B} - Problem 4\\
\rule{500pt}{1pt}\\
\end{flushleft} 

\noindent Part a) $\partial A = \overline{A} \cap \overline{A^c}$ \\ \newline
\indent Suppose $x \in \partial A$, then by definition every neighborhood contains an element of $A$ and $A^c$. This means that x is also a limit point of $A$ and $A^c$ so we have $x \in \overline{A}$ and $x \in \overline{A^c}$. So we have $x \in A \cap A^c$ therefore $\partial A \subset \overline{A} \cap \overline{A^c}$.\\
\indent Now suppose that $x \in \overline{A} \cap \overline{A^c}$. This means that x is a limit point of both $A$ and $A^c$. By the definition of the limit point this means that every neighborhood of x contains a point of $A$ and $A^c$ which is to say that $x \in \partial A$. \newline

\noindent Part b) \newline

\noindent Part c) Show $\partial A$ is closed. \\ \newline
\indent By part a, $\partial A = \overline{A} \cap \overline{A^c}$. By a theorem of rudin the closure of a set is closed. So $\partial A$ is the intersection of two closed sets. By another theorem of rudin, the intersection of two or more closed sets is closed thus $\partial A$ is closed. \newline

\noindent Part d) Show, $A$ is closed $\iff \partial A \subset A$ \\

Sheeeeeeeeeit

\vspace{15pt}
\begin{flushleft} 
\textbf{Class 18.100B} - Problem 5\\
\rule{500pt}{1pt}\\
\end{flushleft} 

Merge.

\vspace{15pt}
\begin{flushleft} 
\textbf{Class 18.100B} - Problem 6\\
\rule{500pt}{1pt}\\
\end{flushleft} 

We need to show that the set $K = \{ 0,1,\frac{1}{2}, \frac{1}{3}, \ldots, \frac{1}{n}, \ldots \}$ is compact using the definition. Suppose $U_\alpha$, a collection of open sets, is a cover of $K$. For some $\alpha_1$ we have 0 $\in U_{\alpha_1}$. Since $U_{\alpha_1}$ is open we know that there is  a neighborhood around 0, $N_l(0)$ with length $l > 0$ such that $N_l(0) \subset U_{\alpha_i}$. Since $l > 0$ and $\mathbb{N}$ is unbounded there must be a $N$ such that $n > N \ge \frac{1}{l}$, where $N - 1 < \frac{1}{l}$. This is to say $l > \frac{1}{n}$ for $n > N$. So, $N_l(0)$ contains 0, and all members of the set except $K _N = \{1, \frac{1}{2}, \ldots, \frac{1}{N}\}$. Because this set is finite and $K$ is covered by $U_\alpha$ there are a finite number of $\alpha_j$ such that $x \in K_N$ and $x \in U_{\alpha_j}$.  Therefore $U_{\alpha_1}$ and the finite number of $U_{\alpha_j}$ are a finite subcover of K and K is compact.

\vspace{15pt}
\begin{flushleft} 
\textbf{Class 18.100B} - Problem 7\\
\rule{500pt}{1pt}\\
\end{flushleft} 

%We would like to find a $\delta$ such that for any $x \in K$,  a compact subset of a metric space, the ball $B_\delta(x) \subset U_\alpha$ for some $U_\alpha \in \{U_\alpha\}_{\alpha \in I}$, an open cover of $K$.\\
%\indent By hypothesis $\{U_\alpha\}_{\alpha \in I}$ is an open cover of K so for all $x \in K$ we have $x \in U_{\alpha_i}$ for some $i \in I$. Since each set in the open cover is open there is an open ball around $x, B(x)$,  such that $B(x) \subset U_{\alpha_i}$. Because there is an open ball around every member of $K$ this constitutes an open cover of $K$ as well. By the compactness of $K$ there is a finite subset of the open balls that cover the set, $\{ B_{\delta_1}(x_1), B_{\delta_2}(x_2), \ldots, B_{\delta_n}(x_n) \}$.

\vspace{15pt}
\begin{flushleft} 
\textbf{Class 18.100B} - Extra Problem 1\\
\rule{500pt}{1pt}\\
\end{flushleft} 
Show that a nonempty set M with d having the following properties is a metric space: \\
i) $d(x,y) = 0 \iff x = y$\\
ii) $d(x,y) \le d(x,z) + d(y,z)$ \\ \newline
\indent First we show that $d(x,y) > 0$ if $x \neq y$ which will fill out the first property of a metric space. By property (i) if $x \neq y$ then $d(x,y) \neq 0$. Assuming $x \neq y$ and using property (ii) if we let $y = x$ and $z = y$ then we get $0 = d(x,x) \le d(x,y) + d(x, y) \implies 0 \le 2 \,d(x,y) \implies 0 \le d(x,y)$. Since we have $d(x,y) \neq 0$ if $x \neq y$ we have $d(x,y) > 0$.\\
\indent For the second property, the symmetry of the metric we use property (ii) by letting $z = x$ to get $d(x,y) \le d(x,x) + d(y,x) \implies d(x,y) \le d(y,x)$ since by property (i) $d(x,x) = 0$. Using property (ii) again we let $x = y$, $y = x$ and $z = x$ and derive $d(y,x) \le d(y,x) + d(x,x) \implies d(y,x) \le d(x,y)$. So we have $d(y,x) \le d(x,y)$ and $d(x,y) \le d(y,x)$ which means $d(x,y) = d(y,x)$ which establishes the second property of the metric.\\
\indent With symmetry established we show the third property (the triangle inequality) easily. From property (ii) and symmetry we have $d(x,y) \le d(x,z) + d(y,z) \implies d(x,y) \le d(x,z) + d(z,y)$ which is the desired inequality.



\end{document}  