%\documentclass[11pt,reqno]{amsart}
\documentclass[11pt,reqno]{article}
\usepackage[margin=.8in, paperwidth=8.5in, paperheight=11in]{geometry}
%\usepackage{geometry}                % See geometry.pdf to learn the layout options. There are lots.
%\geometry{letterpaper}                   % ... or a4paper or a5paper or ... 
%\geometry{landscape}                % Activate for for rotated page geometry
%\usepackage[parfill]{parskip}    % Activate to begin paragraphs with an empty line rather than an indent7
\usepackage{graphicx}
\usepackage{pstricks}
\usepackage{amssymb}
\usepackage{epstopdf}
\usepackage{amsmath}
\usepackage{subfigure}
\usepackage{caption}
\pagestyle{plain}
%\renewcommand{\topfraction}{0.3}
%\renewcommand{\bottomfraction}{0.8}
%\renewcommand{\textfraction}{0.07}
\DeclareGraphicsRule{.tif}{png}{.png}{`convert #1 `dirname #1`/`basename #1 .tif`.png}

\title{Real Analysis $\mathbb{I}$: \\ Assignment 3}
\author{Andrew Rickert}
\date{Started: March 13, 2011 \\ \hspace{1pt} Ended: March ??  2010}                                           % Activate to display a given date or no date

\begin{document}
\maketitle


% Page 1
\begin{flushleft} 
\textbf{Class 18.100B} - Problem 1\\
\rule{500pt}{1pt}\\
\end{flushleft} 

We need to show how the norm can define a metric on $\mathbb{R}$. Suppose we are given the definition of the norm as in the problem. We define the metric to based on the norm as $d(p,q) \equiv \|p-q\|$.\\
\indent We know need to verify that the metric defined in this way satisfies the axioms for a metric.
First we need to show that 
\begin{equation}
d(p,q) > 0 \; \text{if} \; p \neq q \; \text{and} \; d(p,p) = 0 \label{rule1}
\end{equation} 
From rule i) for the norm we know that $d(p,q) = \|p-q\| \ge 0$,  since also we have that $\|x\| = 0 \iff x = 0$ we know that $p-q \ne 0 \implies \|p-q\| \neq 0$. So, if $p \neq q$ then $d(p,q) \neq 0$ which shows $d(p,q) > 0$ and completes the first part of (\ref{rule1}). From the definition of the metric we have $d(p,p) = \|p - p\| = \| 0 \| = 0$ again by rule (i). The completes the second part of (\ref{rule1}).\\
\indent Next we have to show that 
\begin{equation}
d(p,q) = d(q,p) \label{rule2}
\end{equation} 
According to rule (ii) it is true that $\| \lambda x \| = |\lambda| \|x \|$. We perform the following calculation.
\begin{eqnarray*}
d(p,q) & = & \| p - q \| \\
            & = & \| (-1) (q - p) \| \\
            & = & |-1| \|q - p\| \; \text{by rule (ii)} \\
            & = & \| q - p \| = d(q,p)
\end{eqnarray*}
We are given rule (iii) as $\| x+ y \| =  \| x \|  +  \| y \|$. The calculation to verify that $d(p,q) \le d(p,r) + d(r,q)$ goes as follows:
\begin{eqnarray*}
d(p,q) & = & \| p - q \| \\
	  & = & \| p - r + r - q \| \\
	  & = & \| p - r \| + \| r - q \| \; \text{by rule (iii)} \\
	  & = & d(p,r) + d(r,q)
\end{eqnarray*}
This verifies all of the properties of the metric based on the norm defined in the problem.

\newpage
\vspace{15pt}
\begin{flushleft} 
\textbf{Class 18.100B} - Problem 2\\
\rule{500pt}{1pt}\\
\end{flushleft} 

In order to show that $d_1(x,y) = \frac{d(x,y)}{1 + d(x,y)}$ is a metric, given that $d(x,y)$ is, we will use the definition of $d_1(x,y)$ to show the three properties of the metric are satisfied.\\
\indent First we need to $d_1(x,y) > 0$ if $x \neq y$ and $d(x,x) = 0$ otherwise. Now since $d(x,y) > 0$ if $x \neq y$ then $1 + d(x,y) > 1 > 0$ so $1+ d(x,y) > 0$ therefore we have $\frac{d(x,y)}{1 + d(x,y)} > 0$ by the field axioms. Also, because $d(x,x) = 0$ by definition, $d_1(x,x) = \frac{d(x,y)}{1 + d(x,y)} = \frac{0}{1 + 0} = 0$.\\
\indent Next we show that $d_1(x,y)$ is symmetric. This is clear from the following calculation
\begin{eqnarray*}
d_1(x,y) & = & \frac{d(x,y)}{1 + d(x,y)} \\
	      & = & \frac{d(y,x)}{1 + d(y,x)} \quad \text{By the symmetry of $d(x,y)$} \\
	      & = & d_1(y,x)
\end{eqnarray*}
Finally we need to show that $d_1(x,y)$ satisfies the triangle inequality. From the definition of the metric $d$ we have that $d(x,z) \le d(x,y) + d(y,z)$ and similarly $1+ d(x,z) \le 1+ d(x,y) + d(y,z)$. Because $d(p,q) > 0$ for all $p, q$we can divide the first expression by second to get\\
\begin{eqnarray}
d_1(x,y) & = & \frac{d(x,y)}{1 + d(x,y)} \le \frac{d(x,y)+ d(y,z)}{1+ d(x,y) + d(y,z)} \nonumber \\ 
	      & = & \frac{d(x,y)}{1+ d(x,y) + d(y,z)} + \frac{d(y,z)}{1+ d(x,y) + d(y,z)} \label{largeTriangle}
\end{eqnarray}
Now both $0 \le d(x,y)$ and $0 \le d(y,z)$ from the definition of $d(p,q)$ so we have in the first instance  $0 \le d(x,y) \implies 1 + d(y,z) \le 1 + d(x,y) + d(y,z) \implies \frac{1}{1 + d(x,y) + d(y,z)} \le \frac{1}{1 + d(y,z)}$ and by a similar calculation we have $ \frac{1}{1 + d(x,y) + d(y,z)} \le \frac{1}{1 + d(x,y)}$. Using these expressions in (\ref{largeTriangle}) we get the following 
\begin{eqnarray*}
d_1(x,y) & \le & \frac{d(x,y)}{1+ d(x,y) + d(y,z)} + \frac{d(y,z)}{1+ d(x,y) + d(y,z)}  \\
	      & \le  &  \frac{d(x,y)}{1+ d(x,y)} +  \frac{d(y,z)}{1+ d(y,z)} \\
	      & = & d_1(x,y) + d_1(y,z)
\end{eqnarray*}
This completes the demonstration of the third property of the metric.

\vspace{15pt}
\begin{flushleft} 
\textbf{Class 18.100B} - Problem 3\\
\rule{500pt}{1pt}\\
\end{flushleft} 

 
\vspace{15pt}
\begin{flushleft} 
\textbf{Class 18.100B} - Problem 4\\
\rule{500pt}{1pt}\\
\end{flushleft} 


\vspace{15pt}
\begin{flushleft} 
\textbf{Class 18.100B} - Problem 5\\
\rule{500pt}{1pt}\\
\end{flushleft} 


\vspace{15pt}
\begin{flushleft} 
\textbf{Class 18.100B} - Problem 6\\
\rule{500pt}{1pt}\\
\end{flushleft} 


\vspace{15pt}
\begin{flushleft} 
\textbf{Class 18.100B} - Problem 7\\
\rule{500pt}{1pt}\\
\end{flushleft} 


\end{document}  